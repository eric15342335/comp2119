\documentclass{article}
\usepackage{amsmath, amssymb, amsthm}
\usepackage[utf8]{inputenc}

\begin{document}

\section*{Example}
One can easily prove (by M.I.) that
\begin{equation}
   f(n) = \frac{n(n+1)(2n+1)}{6}, \forall n >= 0
\end{equation}

\section*{Proof}

\begin{flushleft}
Let $f(n)$ be "$0^2+1^2+2^2+3^2+...+n^2=\frac{n(n+1)(2n+1)}{6}$".
For $\forall n \geq 0$,
\end{flushleft}

For $n=0$,
\begin{align*}
   & \text{L.H.S.} = 0 \\
   & \text{R.H.S.} = 0 \\
\end{align*}

$\because$ L.H.S. = R.H.S.

$\therefore f(0)$ is true.

\begin{flushleft}

   Assume $S(n)$ is true for some $n=k$ where $k \geq 0$,
   i.e. $0^2+1^2+2^2+3^2+...+k^2=\frac{k(k+1)(2k+1)}{6}$

\end{flushleft}

For $n=k+1$,
\begin{align*}
\text{L.H.S.} &= 0^2+1^2+2^2+3^2+...+k^2+(k+1)^2 \\
&= \frac{k(k+1)(2k+1)}{6}+(k+1)^2 
\text{(By induction assumption)} \\
&= \frac{k(k+1)(2k+1)+6(k+1)^2}{6} \\
&= \frac{(k+1)(k(2k+1)+6(k+1))}{6} \\
&= \frac{(k+1)(2k^2+7k+6)}{6} \\
&= \frac{(k+1)(k+2)(2k+3)}{6} \\
&= \frac{(k+1)((k+1)+1)(2(k+1)+1)}{6} \\
&= \text{R.H.S.}
\end{align*}

\begin{flushleft}
   $\because$ L.H.S. = R.H.S. \\
   $\therefore$ By the principle of mathematical induction,
   $f(n)$ is true for all $n \geq 0$.
\end{flushleft}

\end{document}
