\documentclass{article}
\usepackage{amssymb, amsthm, enumitem, microtype}
\usepackage[utf8]{inputenc}
\usepackage[margin=1in,top=0.6in,bottom=0.6in]{geometry}
\usepackage[fleqn]{amsmath}

\usepackage{verbatim}
\usepackage{minted}
% Add "-shell-escape", to vscode:
% Open settings, search for "latex-workshop.latex.tools", click "Edit in settings.json", add "-shell-escape" to the pdflatex command.

\title{COMP2119 Introduction to Data Structures and Algorithms
Assignment 2 - Algorithm Analysis, Data Structures and Hashing}
\author{Cheng Ho Ming, Eric (3036216734) [Section 1C, 2024]}

\begin{document}
\maketitle

\begin{enumerate}
\item
    \begin{enumerate}[label=(\alph*)]
        \item
        Jeff's code perform a linear search on the list of integers $A$ and finds out the index $k$ of the first element on the list which disrupts the ascending order of the list. \\
        He first initializes an variable $k$, which is a integer that stores the index of the first element that disrupts the ascending order of the list.

        Then, he uses an for-loop with counter $i$ to iterate through the list of integers $A$ from the first element (with index $0$) to the second last element (with index $n-2$). During each iteration, the program compares the current element ($i$) with the next element ($i+1$). If the current element is greater than the next element, he assigns the index of the current element to $k$.

        Finally, he returns the value of $k$.

        Let the cost of the k-th line of the code section above be $c_k$. The worst-case time complexity will be:
        \[
        O(c_3+c_4+(n-2+1)(c_5+c_6)) = O(n)
        \]

        \item The pseudocode of another algorithm $reOrder2(A,n)$ is as follows:
        \begin{minted}[samepage]{pascal}
function reOrder2(A, n)
    left = 0
    right = n-1
    while (left < right - 1) do
        middle = (left+right) div 2 // integer division
        if A[left] < A[middle] then
            left = middle
        end if
        if A[middle] < A[right] then
            right = middle
        end if
    end while
    return right
end function
        \end{minted}
        The code performs a binary search over the list of integers $A$.

        The list $A$ can be divided into two separate ascending sequences of numbers. The code will find the index of the last element of the first sequence, store it in variable $left$, and find the index of the first element of the second sequence and store it in variable $right$.

        \begin{tabular}{@{}l@{\hspace{1em}}l@{}}
            It does this by selecting the element in the middle of the list $A$, and determining whether & \dotfill \\
            such element belongs to the first sequence or the second sequence. & \dotfill \\
            \quad * If the element belongs to the first sequence, & \dotfill \\ 
            \quad \quad it updates the value of $left$ to the index of the middle element. & \dotfill \\
            \quad * If the element belongs to the second sequence, & \dotfill \\ 
            \quad \quad it updates the value of $right$ to the index of the middle element. & \dotfill \\
        \end{tabular}
            

        Ultimately, during each selection, the element selected must belong to either one of the sequences, given the list $A$ is valid.

        The maximum number of times we can select the middle element is $\log_2(n)$, before we reach the point that $left$ and $right$ are adjacent to each other. Therefore, the worst case time complexity will be:
        \[
        O(c_2+c_3+c_4+(\log_2(n))(c_5+c_6+c_7+c_9+c_10)+c_13) = O(\log_2(n)) = O(\log(n))
        \]
    \end{enumerate}

\item
    \begin{enumerate}[label=(\alph*)]
        \item
        \begin{minted}[samepage]{c}
#include <stdlib.h>

struct Node {
    int data;
    struct Node* next;
};
typedef struct Node Node;

Node* mergeTwoSortedLinkedList(Node* first, Node* second) {
    if (first == NULL) return second;
    if (second == NULL) return first;
    if (first->data < second->data) {
        first->next = mergeTwoSortedLinkedList(first->next, second);
        return first;
    }
    else {
        second->next = mergeTwoSortedLinkedList(first, second->next);
        return second;
    }
}
        \end{minted}

        \item
        \begin{minted}[samepage]{c}
#include <stdlib.h>

struct Node {
    int data;
    struct Node* next;
};
typedef struct Node Node;

Node* getCycleBeginsPtr(Node* head) {
    if (head == NULL) return NULL;
    Node* one_step = head;
    Node* two_step = head;
    while (two_step != NULL && two_step->next != NULL) {
        one_step = one_step->next;
        two_step = two_step->next->next;
        if (one_step == two_step) {
            Node* cycle_begins = head;
            while (cycle_begins != one_step) {
                cycle_begins = cycle_begins->next;
                one_step = one_step->next;
            }
            return cycle_begins;
        }
    }
    return NULL;
}
        \end{minted}
    \end{enumerate}

\item
    \begin{enumerate}[label=(\alph*)]
        \item
        \begin{minted}[samepage]{cpp}
#include <vector>

#define MAX_VAL 10
class FreqStack {
    private:
        std::vector<int> stack;
        int freq[MAX_VAL] = {0};
        int maxFreq(void) {
            int max = 0;
            for (int i = 0; i < MAX_VAL; i++) {
                if (freq[i] > max) {
                    max = freq[i];
                }
            }
            return max;
        }
    public:
        FreqStack() {
            stack = std::vector<int>();
        }
        void push(int val) {
            stack.push_back(val); // O(1) operation
            freq[val]++;
        }
        int pop(void) {
            int _maxFreq = maxFreq();
            int i = stack.size() - 1;
            while (i > 0) {
                if (freq[stack[i]] == _maxFreq) {
                    freq[stack[i]]--;
                    break;
                }
                i--;
            }
            int _val = stack[i];
            stack.erase(stack.begin() + i); // O(n) operation
            return _val;
        }
};
        \end{minted}

        \item In my implementation, the worst-case time complexity of $fstack.push$ would be $O(1)$, as both the $stack.push\_back$ and $freq[val]++$ operations are $O(1)$ (constant time) operations.

        For the $fstack.pop$ function:
        \begin{minted}[samepage]{cpp}
int pop(void) {
    int _maxFreq = maxFreq();               // O(n) (maxFreq() is an linear search function)
    int i = stack.size() - 1;               // O(1)
    while (i > 0) {                         // O(n)
        if (freq[stack[i]] == _maxFreq) {   // O(1) (O(n) inside a while loop)
            freq[stack[i]]--;               // O(1) (O(n) inside a while loop)
            break;                          // O(1) (O(n) inside a while loop)
        }
        i--;                                // O(1) (O(n) inside a while loop)
    }
    int _val = stack[i];                    // O(1)
    stack.erase(stack.begin() + i);         // O(n) (erase() involves shifting
                                            // elements to fill the blank)
    return _val;                            // O(1)
}
        \end{minted}
        Let the cost of the k-th line of the code section above be $c_k$. The worst case time complexity will be:
        \[
        O(nc_2+c_3+n(c_4+c_5+c_6+c_7+c_9)+c_9+c_{11}+nc_{12}+c_{13}) = O(n)
        \]
    \end{enumerate}

\item
    \begin{enumerate}[label=(\alph*)]
        \item Inserting $\{17,94,86,22,98,79,54,38\}$ to the hash table with size 7, using the hash function $h(x) = x \mod 7$ and collision handling by chaining.
        \begin{minted}[samepage]{python3}
    # Initial state:
    [NIL, NIL, NIL, NIL, NIL, NIL, NIL]
    # Insert 17 at h(17) = 17 mod 7 = 3:
    [NIL, NIL, NIL, [17], NIL, NIL, NIL]
    # Insert 94 at h(94) = 94 mod 7 = 3
    [NIL, NIL, NIL, [17,94], NIL, NIL, NIL]
    # Insert 86 at h(86) = 86 mod 7 = 2
    [NIL, NIL, [86], [17,94], NIL, NIL, NIL]
    # Insert 22 at h(22) = 22 mod 7 = 1
    [NIL, [22], [86], [17,94], NIL, NIL, NIL]
    # Insert 98 at h(98) = 98 mod 7 = 0
    [[98], [22], [86], [17,94], NIL, NIL, NIL]
    # Insert 79 at h(79) = 79 mod 7 = 2
    [[98], [22], [86,79], [17,94], NIL, NIL, NIL]
    # Insert 54 at h(54) = 54 mod 7 = 5
    [[98], [22], [86,79], [17,94], NIL, [54], NIL]
    # Insert 38 at h(38) = 38 mod 7 = 3
    [[98], [22], [86,79], [17,94,38], NIL, [54], NIL]
        \end{minted}
        As an empty slot or a NIL pointer access should also be counted as one inspection, an unsuccessful search on $h(?) = 0$ would take 2 inspections (1 for accessing 98 and 1 for NIL pointer access), for example. \\
        Average number of slots inspected for an unsuccessful search after the keys are inserted would be:
        \[
        \frac{2+2+3+4+1+2+1}{7} = \frac{15}{7} \approx 2.14 \text{ (correct to 3 significant figures)}
        \]

        \item After insertion, the hash table would look like this:
        \begin{minted}[samepage]{python3}
    [22,86,79,NIL,NIL,54,17,NIL,38,94,98]
        \end{minted}
        The average number of slots inspected for an unsuccessful search after the keys are inserted would be:
        \[
        \frac{2+2+5+1+1+4+4+1+3+3+6}{11} = \frac{32}{11} \approx 2.91 \text{ (correct to 3 significant figures)}
        \]

        \item
        After insertion, the hash table would look like this:
        \begin{minted}[samepage]{python3}
    [22,54,79,NIL,NIL,38,17,NIL,94,86,98]
        \end{minted}
        Since it is a double hash function, the probing sequence can be different even if the first inspected slot is the same.

        In this case, we notice that for $x = 0$ or $11$, their first inspected slots are both $0$ (as $h(0,0) = h(11,0) = 0$), but their next inspected slots are $h(0,1)=2$ and $h(11,1)=1$ respectively.

        Looking at the function, we can see the second hash function $h_2(x) = 2-(x \mod 2)$ returns $2$ if $x$ is even and $1$ if $x$ is odd.

        Since we are calculating average, we assume $x$ has equal chance ($50\%$) for being even or odd.

        The average number of slots inspected for an unsuccessful search after the keys are inserted would be:
        \begin{align*}
        &\quad \ \frac{\frac{4+3}{2}+\frac{3+2}{2}+\frac{2+2}{2}+1+1+\frac{3+2}{2}+\frac{2+5}{2}+1+\frac{7+4}{2}+\frac{6+4}{2}+\frac{5+3}{2}}{11} \\
        &= \frac{63}{22} \approx 2.86 \text{ (correct to 3 significant figures)}
        \end{align*}
    \end{enumerate}

\end{enumerate}

P.S. (Dear TA) I am sorry that copying the code from the pdf does not preserve indentation. Do you know any workarounds for this? I am using $\backslash usepackage\{minted\}$ for code snippets.

\end{document}
