\documentclass{article}
\usepackage{amssymb, amsthm, enumitem}
\usepackage[utf8]{inputenc}
\usepackage[margin=1in,top=0.6in,bottom=0.6in]{geometry}
\usepackage[fleqn]{amsmath}

% \usepackage{minted}
% Add "-shell-escape", to vscode:
% Open settings, search for "latex-workshop.latex.tools", click "Edit in settings.json", add "-shell-escape" to the pdflatex command.

\title{COMP2119 Introduction to Data Structures and Algorithms
Assignment 1 - Recursion, Mathematical Induction and Algorithm Analysis}
\author{Cheng Ho Ming, Eric (3036216734) [Section 1C, 2024]}

\begin{document}
\maketitle

\section{Asymptotic Bounds $[20\%]$}

$
\Theta(\log n): (d)\ {\log n}, (j)\ {\log {\frac{n}{\log n}}}, \\
\Theta(\frac{n}{\log n}): (h)\ \frac{n}{\log n}, \\\
\Theta(\frac{n}{\log {\log n}}): (i)\ \frac{n}{\log {\log n}}, \\
\Theta(n): (g)\ \frac{n}{\log\pi}, (m)\ \sqrt[]{\sum_{i=1}^{n} (i+1)}, \\
\Theta(n^{\log \pi}): (e)\ \pi ^ {\log n}, (f)\ n^{\log \pi}, (k)\ \pi ^ {\log (2n)}, \\
\Theta(n^{\log{2\pi}}): (l)\ n^{\log{2\pi}}, \\
\Theta(n^\pi): (a)\ n^\pi, \\
\Theta(\pi^n): (b)\ \pi^n, \\
\Theta(n^n): (n)\ 1910n! + 316n^n, (c)\ n^n
$

\section{Recurrence Relations $[20\%]$}

\begin{enumerate}[label=(\alph*)]
\item $\begin{aligned}[t]
T(n) &= T(n - 1) + 3 \quad \text{for}\ n > 0\\
&= T(n - 2) + 3 + 3 \quad \text{for}\ n > 1\\
&= \dots \\
\therefore T(n) &= T(0) + 3n\\
&= 3n
\end{aligned}$

$
\therefore T(n) = \Theta(n)$

\item $\begin{aligned}[t]
\text{Assume} & \text{ that n is a power of 3, i.e. $n=3^k$ for $k \in \mathbb{N}$, and $\log_3 n = k$,}\\
\because T(n) &= 3T(\frac{n}{3}) + n \quad \text{for $n\neq1$}\\
&= 3(3T(\frac{n}{9}) + \frac{n}{3}) + n \quad \text{for $n>=9$}\\
&= 9T(\frac{n}{9}) + n + n \quad \text{for $n>=9$}\\
&= ...&\\
\therefore T(n) &= 3^kT(\frac{n}{3^k}) + k*n \\
&= 3^kT(1) + k*n\\
&= 0 + k*n\\
&= kn\\
&= n * \log_3 n
\end{aligned}$

$\ \therefore T(n) = \Theta(n \log n)$

\item $\begin{aligned}[t]
\text{Assume} & \text{ that n is a power of 3, i.e. $n=3^k$ for $k \in \mathbb{N}$, and $\log_3 n = k$,}\\
\because T(n) &= 4T(\frac{n}{3})+1 \quad \text{for $n>=3$}\\
&= 4(4T(\frac{n}{9}) + 1) + 1 \quad \text{for $n>=9$}\\
&= 16T(\frac{n}{9}) + 4^{k-1} + ... + 4^0 \quad \text{for $n>=9$}\\
&= ...\\
\therefore T(n) &= 4^kT(\frac{n}{3^k}) + \frac{1*(1-4^k)}{1-4} \\
&= 4^kT(1) + \frac{4^k-1}{3} \\
&= 0 + \frac{4^{log_3 n}-1}{3} \\
&= \frac{1}{3} * 4^{log_3 n} - \frac{1}{3} \\
&= \frac{1}{3} * 3^{(log_3 4)(log_3 n)} - \frac{1}{3} \\
&= \frac{1}{3} * n^{log_3 4} - \frac{1}{3} \\
\end{aligned}$

$\ \therefore T(n) = \Theta(n^{log_3 4})$

\item $\begin{aligned}[t]
\text{Assume} & \text{ that n is a power of 2, i.e. $n=2^k$ for $k \in \mathbb{N}$, and $\log_2 n = k$,}\\
\because T(n) &= nT(\frac{n}{2})+n-1 \quad \text{for $n>=2$}\\
&= n*(\frac{n}{2}*T(\frac{n}{4}) + \frac{n}{2} - 1) + n - 1 \quad \text{for $n>=4$}\\
&= \frac{n^2}{2}*T(\frac{n}{4}) + \frac{n^2}{2} - n + n - 1 \quad \text{for $n>=4$}\\
&= \frac{n^2}{2}*T(\frac{n}{4}) + \frac{n^2}{2} - 1 \quad \text{for $n>=4$}\\
&= \frac{n^2}{2}*(\frac{n}{4}*T(\frac{n}{8}) + \frac{n}{4} - 1) + \frac{n^2}{2} - 1 \quad \text{for $n>=8$}\\
&= \frac{n^3}{2*4}*T(\frac{n}{8})+\frac{n^3}{2*4} - \frac{n^2}{2} + \frac{n^2}{2} - 1 \quad \text{for $n>=8$}\\
&= \frac{n^3}{2^1*2^2}*T(\frac{n}{2^1*2^2})+\frac{n^3}{2^1*2^2} - 1 \quad \text{for $n>=8$}\\
&= \dots\\
\therefore T(n) &= \frac{n^k}{2^0*2^1*2^2*...*2^{k-1}}*T(1) + \frac{n^k}{2^0*2^1*2^2*...*2^{k-1}} - 1\\
&= \frac{n^k}{2^{\frac{(k-1)*k}{2}}}*2-1\\
&= \frac{n^k}{n^{\frac{k-1}{2}}}*2-1\\
&= n^{\frac{2k}{2}-\frac{k-1}{2}}*2-1\\
&= n^{\frac{k+1}{2}}*2-1\\
&= 2n^{\frac{log_2 n +1}{2}}-1\\
\end{aligned}$

$\ \therefore T(n) = \Theta(n^{\frac{log_2 n+1}{2}})$

\end{enumerate}

\section{Mathematical Induction $[30\%]$}

\begin{enumerate}[label=(\alph*)]

\item $\text{Let } f(n) \text{ be the predicate } "1*2^1 + 2*2^2 + 3*2^3 + ... + n*2^n = (n-1)2^{n+1} + 2" \text{ for } \forall n \in \mathbb{Z}^+.$

$\begin{aligned}
\text{For } n=1, \text{L.H.S.} &= 1*2^1 \\
&= 2 \\
\text{R.H.S.} &= (1-1)2^{1+1} + 2 \\
&= 2
\end{aligned}$

$\because$ L.H.S. = R.H.S.

$\therefore f(1)$ is true.

\paragraph*{Inductive Step:\\}
Assume that $f(n)$ is true when $n = k$, for some $k \in \mathbb{Z}^+$, i.e. $f(k) = 1*2^1 + 2*2^2 + 3*2^3 + ... + k*2^k = (k-1)2^{k+1} + 2$.

$\begin{aligned}
\text{Consider the case } n=k+1, \ \text{L.H.S.} &= 1*2^1 + 2*2^2 + 3*2^3 + ... + k*2^k + (k+1)*2^{k+1} \\
&= (k-1)2^{k+1} + 2 + (k+1)*2^{k+1} \quad \text{(by induction hypothesis)} \\
&= 2^{k+1} * (2k) + 2 \\
&= (k)2^{k+2} + 2 \\
&= (k+1-1)2^{k+1+1} + 2 \\
&= \text{R.H.S.}
\end{aligned}$

$\because f(n)$ is true when $n = k+1$.

$\therefore$ By the principle of mathematical induction,
$f(n)$ is true for all $n \in \mathbb{Z}^+$.

\item Let $f(n)$ be the predicate "$\frac{1}{\sqrt{1}+\sqrt{2}} + \frac{1}{\sqrt{2}+\sqrt{3}} + \frac{1}{\sqrt{3}+\sqrt{4}} + ... + \frac{1}{\sqrt{n}+\sqrt{n+1}} = \sqrt{n+1}-1$ for $\forall n \in \mathbb{Z}^+$ ".

$\begin{aligned}
\text{For } n=1, \text{L.H.S.} &= \frac{1}{\sqrt{1}+\sqrt{2}} \\
&= \frac{1}{\sqrt{1}+\sqrt{2}} * \frac{\sqrt{1}-\sqrt{2}}{\sqrt{1}-\sqrt{2}} \\
&= \frac{\sqrt{1}-\sqrt{2}}{1-2} \\
&= \sqrt{2} - 1 \\
\text{R.H.S.} &= \sqrt{1+1} - 1 \\
&= \sqrt{2} - 1 \\
\because \text{L.H.S.} = \text{R.H.S.} \\
\therefore f(1) \text{ is true.}
\end{aligned}$

\paragraph*{Inductive Step:\\}
Assume that $f(n)$ is true when $n = k$, for some $k \in \mathbb{Z}^+$, i.e. $f(k) = \frac{1}{\sqrt{1}+\sqrt{2}} + \frac{1}{\sqrt{2}+\sqrt{3}} + \frac{1}{\sqrt{3}+\sqrt{4}} + ... + \frac{1}{\sqrt{k}+\sqrt{k+1}} = \sqrt{k+1}-1$.

Consider the case $n=k+1$,

$\begin{aligned}
\text{L.H.S.} &= \frac{1}{\sqrt{1}+\sqrt{2}} + \frac{1}{\sqrt{2}+\sqrt{3}} + \frac{1}{\sqrt{3}+\sqrt{4}} + ... + \frac{1}{\sqrt{k}+\sqrt{k+1}} + \frac{1}{\sqrt{k+1}+\sqrt{k+2}} \\
&= \sqrt{k+1}-1 + \frac{1}{\sqrt{k+1}+\sqrt{k+2}} \quad \text{(by induction hypothesis)} \\
&= \sqrt{k+1}-1 + \frac{1}{\sqrt{k+1}+\sqrt{k+2}} * \frac{\sqrt{k+1}-\sqrt{k+2}}{\sqrt{k+1}-\sqrt{k+2}} \\
&= \sqrt{k+1}-1 + \frac{\sqrt{k+1}-\sqrt{k+2}}{1-2} \\
&= \sqrt{k+1}-1 + \sqrt{k+2} - \sqrt{k+1} \\
&= \sqrt{k+2} - 1 \\
&= \sqrt{k+1+1} - 1 \\
&= \text{R.H.S.}
\end{aligned}$

$\because f(n)$ is true when $n = k+1$.

$\therefore$ By the principle of mathematical induction,
$f(n)$ is true for all $n \in \mathbb{Z}^+$.

\end{enumerate}

\section{Algorithm Design $[30\%]$}

The algorithm is: \\
For $\ n \in \mathbb{Z}^+$,
\begin{align*}
f(n) = 2^n = \begin{cases}
    2, & \text{if } n = 1, \\
    f(\frac{n}{2})^2, & \text{if } n \text{ is even}, \\
    2 * f(n-1), & \text{if } n \text{ is odd and } n>1.
\end{cases}
\end{align*}
Correctness of the algorithm:

\paragraph*{Base case:}
\begin{align*}
\text{For } n &= 1, f(1) = 2^1 = 2, \text{ which is correct.}
\end{align*}

\paragraph*{Inductive Step:}
\begin{align*}
\text{For } n &= 2k, \text{ where } k \in \mathbb{Z}^+ \\
f(2k) &= f(k)^2 = (2^k)^2 = 2^{2k} = 2^{2k}, \text{ which is correct.}
\end{align*}
\begin{align*}
\text{For } n = 2&k+1, \text{ where } k \in \mathbb{Z}^+ \\
f(2k+1) &= 2 * f(2k+1-1) = 2 * f(2k) \\
&= 2 * f(k)^2 \\
&= 2 * 2^{2k} \\
&= 2^{2k+1}, \text{ which is correct.}
\end{align*}

$\because$ f(2k) is true if f(k) is true, and f(2k+1) is true if f(2k) is true.

$\therefore$ By the principle of mathematical induction, the algorithm is correct.

\text{ } \\
Running time of the algorithm:
\begin{align*}
f(n) &= f(\frac{n}{2})^2, \text{ if n is even} \\
&= f(\frac{n}{4})^{2^2}, \text{ if } \frac{n}{2} \text{ is even} \\
&= \dots \\
&= f(\frac{n}{2^k})^{2^{2^{2^{\dots}} \text{ <-- k amount of "2"s}}}, \text{ for } n = 2^k \text{ such that } k = \log_2 n \\
&= 2^{2^{2^{\dots}}}
\end{align*}
$\because$ in this case, our algorithm needs to perform $k = \log_2 n$ squaring operations, and each squaring operation takes $O(1)$ time (as the typical algorithm for squaring a number: $2^n = 2 * 2^{n-1}$ for $n \in \mathbb{Z}^+$ and $n>1$ takes $O(n)$ time). \\
$\therefore$ $f(n) = O(\log n)$ if $n$ is even.
\begin{align*}
\text{If } n \text{ is odd and } n>1, f(n) &= 2 * f(n-1) \\
&= 2 * O(f(n - 1)) \text{ where } n - 1 \text{ is an even number, } \\
&= O(f(n - 1)), \text{ which is given to be } O(\log n).
\end{align*}
$\therefore$ the running time of the algorithm is $O(\log n)$, which is faster than $O(n)$.

\end{document}
