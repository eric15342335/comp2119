\documentclass{article}
\usepackage{amssymb, amsthm, enumitem}
\usepackage[utf8]{inputenc}
\usepackage[margin=1in,top=0.6in]{geometry}
\usepackage[fleqn]{amsmath}

\title{COMP2119 Introduction to Data Structures and Algorithms
Assignment 1 - Recursion, Mathematical Induction and Algorithm Analysis}
\author{Cheng Ho Ming, Eric (3036216734)}

\begin{document}
\maketitle

\section{Asymptotic Bounds}

$n^\pi, \pi^n, n^n,{\log n}, \pi ^ {\log n}, n^{\log \pi}, \frac{n}{\log\pi}, \frac{n}{\log n}, \frac{n}{\log {\log n}}, {\log {\frac{n}{\log n}}},
\pi ^ {\log (2n)}, n^{\log{2\pi}}, \sqrt[]{\sum_{i=1}^{n} (i+1)}, 1910n! + 316n^n$

\section{Recurrence Relations}

\begin{enumerate}[label=(\alph*)]
\item $\begin{aligned}[t]
T(n) &= T(n - 1) + 3 \quad \text{for}\ n > 0\\
&= T(n - 2) + 3 + 3 \quad \text{for}\ n > 1\\
&= \dots \\
&= T(0) + 3n \quad \text{for}\ n > 1\\
&= 3n \quad \text{for}\ n > 1
\end{aligned}$

$\ \therefore T(n) = \Theta(n)$

\item $\begin{aligned}[t]
\text{Assume} & \text{ that n is a power of 3, i.e. $n=3^k$ for $k \in \mathbb{N}$, and $\log_3 n = k$,}\\
\because T(n) &= 3T(\frac{n}{3}) + n \quad \text{for $n\neq1$}\\
&= 3(3T(\frac{n}{9}) + \frac{n}{3}) + n \quad \text{for $n>=9$}\\
&= 9T(\frac{n}{9}) + n + n \quad \text{for $n>=9$}\\
&= ...&\\
\therefore T(n) &= 3^kT(\frac{n}{3^k}) + k*n \\
&= 3^kT(1) + k*n\\
&= 0 + k*n\\
&= kn\\
&= n * \log_3 n
\end{aligned}$

$\ \therefore T(n) = \Theta(n \log n)$

\item $\begin{aligned}[t]
\text{Assume} & \text{ that n is a power of 3, i.e. $n=3^k$ for $k \in \mathbb{N}$, and $\log_3 n = k$,}\\
\because T(n) &= 4T(\frac{n}{3})+1 \\
&=
\end{aligned}$

$\ \therefore T(n) = \Theta()$

\item $\begin{aligned}[t]
\text{Assume} & \text{ that n is a power of 2, i.e. $n=2^k$ for $k \in \mathbb{N}$, and $\log_2 n = k$,}\\
\because T(n) &= nT(\frac{n}{2})+n-1\\
&=
\end{aligned}$

$\ \therefore T(n) = \Theta()$

\end{enumerate}

\section{Mathematical Induction}

\begin{enumerate}[label=(\alph*)]

\item $\text{Let } f(n) \text{ be the predicate } "1*2^1 + 2*2^2 + 3*2^3 + ... + n*2^n = (n-1)2^{n+1} + 2" \text{ for } \forall n \in \mathbb{Z}^+.$

$\begin{aligned}
\text{For } n=1, \text{L.H.S.} &= 1*2^1 \\
&= 2 \\
\text{R.H.S.} &= (1-1)2^{1+1} + 2 \\
&= 2
\end{aligned}$

$\because$ L.H.S. = R.H.S.

$\therefore f(1)$ is true.

\end{enumerate}

\section{Algorithm Design}

\end{document}
